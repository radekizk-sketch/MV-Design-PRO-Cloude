
\documentclass[11pt,a4paper]{article}
\usepackage[utf8]{inputenc}
\usepackage[T1]{fontenc}
\usepackage[polish]{babel}
\usepackage{amsmath,amssymb}
\usepackage{booktabs}
\usepackage{geometry}
\usepackage{fancyhdr}
\usepackage{lastpage}
\usepackage{xcolor}

\geometry{margin=2.5cm}

\pagestyle{fancy}
\fancyhf{}
\fancyhead[L]{\textit{MV-DESIGN-PRO — Proof Engine}}
\fancyhead[R]{\textit{Dowód matematyczny}}
\fancyfoot[C]{\thepage\ / \pageref{LastPage}}

\definecolor{resultbox}{RGB}{230,245,230}

\newcommand{\resultbox}[1]{%
  \colorbox{resultbox}{\parbox{0.9\textwidth}{\centering #1}}%
}




\title{Dowód obliczeń zwarciowych IEC 60909 — zwarcie asymetryczne}
\author{MV-DESIGN-PRO}
\date{2026-02-06 10:00:00}


\begin{document}

\maketitle

\tableofcontents

\newpage

\section{Dane wejściowe}
\begin{itemize}
  \item Projekt: Test Project §4.1
  \item Przypadek: Golden Case 1f\_z
  \item Wersja solvera: 1.0.0-test
  \item Miejsce zwarcia: B1
  \item Typ zwarcia: SC1FZ
  \item Współczynnik napięciowy:
$$
c = 1.1000
$$
\end{itemize}





\section{Dowód}

\subsection{Krok 1: Impedancje składowych zgodna/przeciwna/zerowa}

\textbf{Wzór:}
$$
\begin{aligned}Z_1 &= Z_{1,\mathrm{th}} \\Z_2 &= Z_{2,\mathrm{th}} \\Z_0 &= Z_{0,\mathrm{th}}\end{aligned}
$$

\textbf{Dane:}
\begin{itemize}
  \item
$$
Z_0 = 0.8000 + j2.4000\,\text{Ω}
$$
  \item
$$
Z_1 = 0.5000 + j1.2000\,\text{Ω}
$$
  \item
$$
Z_2 = 0.6000 + j1.1000\,\text{Ω}
$$
\end{itemize}

\textbf{Podstawienie:}
$$
Z_1 = 0.5000 + j1.2000\,\text{Ω}, \quad Z_2 = 0.6000 + j1.1000\,\text{Ω}, \quad Z_0 = 0.8000 + j2.4000\,\text{Ω}
$$

\textbf{Wynik:}
\begin{center}
$$
Z_1 = 0.5000 + j1.2000\,\text{Ω}
$$
\end{center}

\textbf{Weryfikacja jednostek:}
$$
Ω = Ω ✓
$$

\subsection{Krok 2: Operator a i macierz Fortescue (składowe → fazy)}

\textbf{Wzór:}
$$
a = e^{j 120^\circ}, \quad a^2 = e^{j 240^\circ} \\\begin{bmatrix} I_a \\ I_b \\ I_c \end{bmatrix} =\begin{bmatrix} 1 & 1 & 1 \\ 1 & a^2 & a \\ 1 & a & a^2 \end{bmatrix}\begin{bmatrix} I_0 \\ I_1 \\ I_2 \end{bmatrix}
$$

\textbf{Dane:}
\begin{itemize}
\end{itemize}

\textbf{Podstawienie:}
$$
a = e^{j 120^\circ} = -0.5000 + j0.8660\,\text{—}
$$

\textbf{Wynik:}
\begin{center}
$$
a = -0.5000 + j0.8660\,\text{—}
$$
\end{center}

\textbf{Weryfikacja jednostek:}
$$
— (bezwymiarowy) ✓
$$

\subsection{Krok 3: Sieć składowych dla zwarcia 1F–Z}

\textbf{Wzór:}
$$
Z_k = Z_1 + Z_2 + Z_0
$$

\textbf{Dane:}
\begin{itemize}
  \item
$$
Z_0 = 0.8000 + j2.4000\,\text{Ω}
$$
  \item
$$
Z_1 = 0.5000 + j1.2000\,\text{Ω}
$$
  \item
$$
Z_2 = 0.6000 + j1.1000\,\text{Ω}
$$
\end{itemize}

\textbf{Podstawienie:}
$$
Z_k = 0.5000 + j1.2000\,\text{Ω} + 0.6000 + j1.1000\,\text{Ω} + 0.8000 + j2.4000\,\text{Ω} = 1.9000 + j4.7000\,\text{Ω}
$$

\textbf{Wynik:}
\begin{center}
$$
Z_k = 1.9000 + j4.7000\,\text{Ω}
$$
\end{center}

\textbf{Weryfikacja jednostek:}
$$
Ω + Ω + Ω = Ω ✓
$$

\subsection{Krok 4: Wyznaczenie prądów składowych I₁, I₂, I₀}

\textbf{Wzór:}
$$
\begin{aligned}\text{1F–Z:} \quad & I_1 = I_2 = I_0 = \frac{U_f}{Z_k} \\\text{2F:} \quad & I_1 = \frac{U_f}{Z_k}, \quad I_2 = -I_1, \quad I_0 = 0 \\\text{2F–Z:} \quad & I_1 = \frac{U_f}{Z_k}, \quadI_2 = -\frac{Z_0}{Z_2 + Z_0} I_1, \quadI_0 = -\frac{Z_2}{Z_2 + Z_0} I_1\end{aligned}
$$

\textbf{Dane:}
\begin{itemize}
  \item
$$
U_f = 9.5263\,\text{kV}
$$
  \item
$$
Z_0 = 0.8000 + j2.4000\,\text{Ω}
$$
  \item
$$
Z_2 = 0.6000 + j1.1000\,\text{Ω}
$$
  \item
$$
Z_k = 1.9000 + j4.7000\,\text{Ω}
$$
\end{itemize}

\textbf{Podstawienie:}
$$
I_1 = I_2 = I_0 = 9.5263 / 1.9000 + j4.7000\,\text{Ω} = 0.7043 - j1.7422\,\text{kA}
$$

\textbf{Wynik:}
\begin{center}
$$
I_1 = 0.7043 - j1.7422\,\text{kA}
$$
\end{center}

\textbf{Weryfikacja jednostek:}
$$
kV / Ω = kA ✓
$$

\subsection{Krok 5: Rekonstrukcja prądów fazowych Ia, Ib, Ic}

\textbf{Wzór:}
$$
\begin{bmatrix} I_a \\ I_b \\ I_c \end{bmatrix} =\begin{bmatrix} 1 & 1 & 1 \\ 1 & a^2 & a \\ 1 & a & a^2 \end{bmatrix}\begin{bmatrix} I_0 \\ I_1 \\ I_2 \end{bmatrix}
$$

\textbf{Dane:}
\begin{itemize}
  \item
$$
I_0 = 0.7043 - j1.7422\,\text{kA}
$$
  \item
$$
I_1 = 0.7043 - j1.7422\,\text{kA}
$$
  \item
$$
I_2 = 0.7043 - j1.7422\,\text{kA}
$$
  \item
$$
a = -0.5000 + j0.8660\,\text{—}
$$
\end{itemize}

\textbf{Podstawienie:}
$$
I_a = I_0 + I_1 + I_2 = 2.1128 - j5.2265\,\text{kA}, \quad I_b = I_0 + a^2 I_1 + a I_2 = 0.0000 - j0.0000\,\text{kA}, \quad I_c = I_0 + a I_1 + a^2 I_2 = 0.0000 - j0.0000\,\text{kA}
$$

\textbf{Wynik:}
\begin{center}
$$
I_a = 2.1128 - j5.2265\,\text{kA}
$$
\end{center}

\textbf{Weryfikacja jednostek:}
$$
kA = kA ✓
$$

\subsection{Krok 6: Początkowy prąd zwarciowy I″k (zwarcie asymetryczne)}

\textbf{Wzór:}
$$
\begin{aligned}\text{1F–Z:} \quad & I_k'' = \frac{\sqrt{3} \cdot c \cdot U_n}{|Z_1 + Z_2 + Z_0|} \\\text{2F:} \quad & I_k'' = \frac{c \cdot U_n}{|Z_1 + Z_2|} \\\text{2F–Z:} \quad & I_k'' = \frac{c \cdot U_n}{|Z_k|}\end{aligned}
$$

\textbf{Dane:}
\begin{itemize}
  \item
$$
U_n = 15.0000\,\text{kV}
$$
  \item
$$
Z_k = 1.9000 + j4.7000\,\text{Ω}
$$
  \item
$$
c = 1.1000\,\text{—}
$$
\end{itemize}

\textbf{Podstawienie:}
$$
I_k'' = \frac{\sqrt{3} \cdot 1.1000 \cdot 15.0000}{|Z_k|} = \frac{1.7321 \cdot 16.5000}{5.0695} = 5.6374\,\text{kA}
$$

\textbf{Wynik:}
\begin{center}
$$
I_k'' = 5.6374\,\text{kA}
$$
\end{center}

\textbf{Weryfikacja jednostek:}
$$
kV / Ω = kA ✓
$$

\subsection{Krok 7: Współczynnik udaru κ (zwarcie asymetryczne)}

\textbf{Wzór:}
$$
\kappa = 1{,}02 + 0{,}98 \cdot e^{-3 \cdot R_k / X_k}
$$

\textbf{Dane:}
\begin{itemize}
  \item
$$
R_k = 1.9000\,\text{Ω}
$$
  \item
$$
X_k = 4.7000\,\text{Ω}
$$
\end{itemize}

\textbf{Podstawienie:}
$$
\kappa = 1.02 + 0.98 \cdot e^{-3 \cdot 0.4043} = 1.02 + 0.98 \cdot 0.2974 = 1.3114
$$

\textbf{Wynik:}
\begin{center}
$$
\kappa = 1.3114\,\text{—}
$$
\end{center}

\textbf{Weryfikacja jednostek:}
$$
— (bezwymiarowy) ✓
$$

\subsection{Krok 8: Prąd udarowy ip (zwarcie asymetryczne)}

\textbf{Wzór:}
$$
i_p = \kappa \cdot \sqrt{2} \cdot I_k''
$$

\textbf{Dane:}
\begin{itemize}
  \item
$$
I_k'' = 5.6374\,\text{kA}
$$
  \item
$$
\kappa = 1.3114\,\text{—}
$$
\end{itemize}

\textbf{Podstawienie:}
$$
i_p = 1.3114 \cdot \sqrt{2} \cdot 5.6374 = 1.3114 \cdot 1.4142 \cdot 5.6374 = 10.4553\,\text{kA}
$$

\textbf{Wynik:}
\begin{center}
$$
i_p = 10.4553\,\text{kA}
$$
\end{center}

\textbf{Weryfikacja jednostek:}
$$
— · — · kA = kA ✓
$$

\subsection{Krok 9: Prąd dynamiczny I\_dyn (zwarcie asymetryczne)}

\textbf{Wzór:}
$$
I_{dyn} = i_p
$$

\textbf{Dane:}
\begin{itemize}
  \item
$$
i_p = 10.4553\,\text{kA}
$$
\end{itemize}

\textbf{Podstawienie:}
$$
I_{dyn} = i_p = 10.4553\,\text{kA}
$$

\textbf{Wynik:}
\begin{center}
$$
I_{dyn} = 10.4553\,\text{kA}
$$
\end{center}

\textbf{Weryfikacja jednostek:}
$$
kA = kA ✓
$$

\subsection{Krok 10: Prąd cieplny równoważny I\_th (zwarcie asymetryczne)}

\textbf{Wzór:}
$$
I_{th} = I_k'' \cdot \sqrt{m + n}
$$

\textbf{Dane:}
\begin{itemize}
  \item
$$
I_k'' = 5.6374\,\text{kA}
$$
  \item
$$
m = 1.0000\,\text{—}
$$
  \item
$$
n = 0.0000\,\text{—}
$$
\end{itemize}

\textbf{Podstawienie:}
$$
I_{th} = 5.6374 \cdot \sqrt{1.0000 + 0.0000} = 5.6374 \cdot 1.0000 = 5.6374\,\text{kA}
$$

\textbf{Wynik:}
\begin{center}
$$
I_{th} = 5.6374\,\text{kA}
$$
\end{center}

\textbf{Weryfikacja jednostek:}
$$
kA · — = kA ✓
$$

\section{Podsumowanie wyników}

\textbf{Wyniki końcowe:}
\begin{itemize}
  \item
$$
I_0 = 0.7043 - j1.7422\,\text{kA}
$$
  \item
$$
I_1 = 0.7043 - j1.7422\,\text{kA}
$$
  \item
$$
I_2 = 0.7043 - j1.7422\,\text{kA}
$$
  \item
$$
I_a = 2.1128 - j5.2265\,\text{kA}
$$
  \item
$$
I_b = 0.0000 + j0.0000\,\text{kA}
$$
  \item
$$
I_c = 0.0000 + j0.0000\,\text{kA}
$$
  \item
$$
I_{dyn} = 10.4553\,\text{kA}
$$
  \item
$$
I_k'' = 5.6374\,\text{kA}
$$
  \item
$$
i_p = 10.4553\,\text{kA}
$$
  \item
$$
I_{th} = 5.6374\,\text{kA}
$$
  \item
$$
\kappa = 1.3114\,\text{—}
$$
  \item
$$
Z_k = 1.9000 + j4.7000\,\text{Ω}
$$
\end{itemize}

Liczba kroków: 10

Weryfikacja jednostek: \textcolor{green}{PASS}

\end{document}